% !TEX encoding = ISO-8859-1

% ---
% Capitulo Genérico
% ---
\chapter[Exemplo de capítulo com um título muito extenso...]
		{Exemplo de capítulo com um título muito extenso em que se explica
		como usar referências bibliográficas}

Este capítulo, cujo título é propositalmente muito extenso, 
apresenta mais alguns comandos específicos da classe \abnTeX\ e 
mostra como incluir referências bibliográficas no texto. O arquivo
\texttt{bib/referencias.bib} contém exemplos dos tipos de referência
mais comuns.

% ---
\section{Divisões do documento: seção}\label{sec-divisoes}
% ---

\LaTeX\ gera a numeração das diversas seções do documento automaticamente. A
classe \abnTeX\ herdou do \LaTeX\ os seguintes comandos para 
dividir o texto em seções, em ordem decrescente de nível:
	\begin{quotation}
	\verb|\part| $\hookrightarrow$ \verb|\chapter| 
	$\hookrightarrow$ \verb|\section| $\hookrightarrow$ \verb|\subsection| 
	$\hookrightarrow$ \\
	$\hookrightarrow$ \verb|\subsubsection| $\hookrightarrow$ \verb|\paragraph| 
	$\hookrightarrow$ \verb|\subparagraph|
	\end{quotation}
Esta seção ilustra o uso de divisões de documentos. \LaTeX\ numera
as seções e subseções até o nível \verb|\subsubsection|. Os níveis
subsequentes não são numerados.

\subsection{Divisões do documento: subseção}\label{subsectA}

Isto é uma subseção. Num texto bem organizado, não é preciso ir além
deste nível de seccionamento numerado, pois a numeração já contém 
três níveis: \ref{subsectA}.

Sed vel dolor a libero dignissim ultrices cursus sed dui. 
Suspendisse sed auctor mi, ac feugiat elit. Aenean in porta lectus, 
nec viverra nisl.


\subsubsection{Divisões do documento: subsubseção}

O nível mais baixo de seções numeradas que você pode usar, no 
\LaTeX, é a \emph{subsubseção}.

  Para encher linguiça, mais texto \emph{dummy}.
  Sed consectetur mauris ipsum, in vehicula lacus viverra vel. 
  Quisque accumsan nulla neque. Nunc dictum mollis dolor ut 
  iaculis. Vivamus aliquam erat nec ante eleifend. 

\subsubsection{Divisões do documento: subsubseção}

Isto é outra subsubseção, o nível mais baixo de seccionamento numerado do texto.
Se você quiser ir além disso, use o comando \verb|\paragraph|.

\paragraph{Subseção de subsubseção}

o comando \verb|\paragraph| não numera a subseção da subsubseção. 

 Para encher linguiça, mais texto \emph{dummy}. Nulla ut nisi 
 vitae tortor molestie pulvinar. Vestibulum sed vehicula nisi. 
 Vivamus sit amet sodales tellus, quis sollicitudin orci.

\subsection{Divisões do documento: subseção}\label{sec-exemplo-subsec}

Isto é uma subseção. Para encher linguiça, mais texto \emph{dummy}. 
Praesent ante mauris, varius blandit pretium eget, blandit ut felis. 
Proin vestibulum ex ac rutrum mattis. Integer ultrices sagittis 
fringilla.

\subsubsection{Divisões do documento: subsubseção}

Isto é mais uma subsubseção da \autoref{sec-exemplo-subsec}.

% ---
\section{Este é um exemplo de nome de seção longo. Ele deve estar
alinhado à esquerda e a segunda e demais linhas devem iniciar logo abaixo da
primeira palavra da primeira linha}
% ---

Isso atende à norma \citeonline[seções de 5.2.2 a 5.2.4]{NBR14724:2011} 
 e \citeonline[seções de 3.1 a 3.8]{NBR6024:2012}.


% ---
\section{Consulte o manual da classe \texttt{abntex2.cls}}
% ---

Consulte o manual da classe \abnTeX\  \cite{abntex2classe} para uma
referência completa das macros e ambientes disponíveis. Além disso, o manual
possui informações adicionais sobre as normas ABNT observadas pelo \abnTeX.
 
% ---
\section{Referências bibliográficas}
% ---

A classe \abnTeX\ usa o pacote \texttt{abntex2cite.sty} para gerar a
bibliografia e as citações bibliográficas ao longo do texto. São dois
os comandos que podem ser usados para fazer citações: 
\verb|\cite{|\emph{rótulo}\verb|}|, para referências implícitas, e 
\verb|\citeonline{|\emph{rótulo}\verb|}|, para referências explícitas.

Uma referência implícita é quando o(s) nome(s) do(s) autor(s) não 
faz(em) parte da sentença, como no exemplo a seguir:\\[6pt]
%\hspace*{1em}%
	{\parbox[t]{6.7cm}{%
	\ttfamily\footnotesize\sloppy
	Valores da aderência \$f\$ variam \\ entre 0,33
	(trilho totalmente limpo \\ e seco) e 0,10, 
	quando o trilho está sujo com óleo 
	\textbackslash{}cite[p.$\sim$82]\{hay82\}.}}
\hspace{\fill} ...que produz...\hspace{\fill}
	{\parbox[t]{5.5cm}{%
	\footnotesize Valores da aderência $f$ variam entre 0,33
	(trilho totalmente limpo e seco) e 0,10, 
	quando o trilho está sujo com óleo 
	\cite[p.~82]{hay82}.}}\\[12pt]
	
Referências implícitas são feitas usando-se o comando 
\verb|\cite{|\emph{rótulo}\verb|}|, em que \emph{rótulo} identifica
o documento bibliográfico a que se faz referência -- no caso do
exemplo, \texttt{hay82}. Os exemplos a seguir mostram casos em que
há mais de um autor:\\[6pt]
	{\parbox[t]{6cm}{%
	\ttfamily\footnotesize
	Uma explicação mais detalhada\\ 
	pode ser encontrada na\\ literatura 
	\textbackslash{}cite[p.$\sim$128]\{kop95\}.}}
\hspace{1em} ...que produz...\hspace{1em}
	{\parbox[t]{6cm}{%
	\footnotesize Uma explicação mais detalhada pode
	ser encontrada na literatura  
	\cite[p.~128]{kop95}.}}\\[12pt]
Quando há três autores, o uso do comando 
\verb|\citeonline{|\emph{rótulo}\verb|}| produz:\\[6pt]
	{\parbox[t]{5.5cm}{%
	\ttfamily\footnotesize
	O fenômeno não ocorre no escuro
	\textbackslash{}cite\{huey03\}.}}
\hspace{1em} ...que produz...\hspace{1em}
	{\parbox[t]{6.5cm}{%
	\footnotesize
	O fenômeno não ocorre no escuro 
	\cite{huey03}}}\\[12pt]
Quando há mais de três autores, \texttt{abntex2cite.sty} usa 
\emph{et~al.}:\\[6pt]
	{\parbox[t]{5.5cm}{%
	\ttfamily\footnotesize
	Esse fenômeno só ocorre no escuro
	\textbackslash{}cite\{huey05\}.}}
\hspace{1em} ...que produz...\hspace{1em}
	{\parbox[t]{6.5cm}{%
	\footnotesize 
	Esse fenômeno só ocorre no escuro
	\cite{huey05}.}}\\[12pt]

Referências explícitas são aquelas em que o(s) nome(s) do(s) autor(s)  
faz(em) parte da sentença. O comando \verb|\citeonline{|\emph{rótulo}\verb|}|
deve ser usado nesses casos, como mostra o exemplo a seguir:\\[6pt]
	{\parbox[t]{5.5cm}{%
	\ttfamily\footnotesize
	Uma explicação mais detalhada\\ 
	pode ser encontrada em\\ 
	\textbackslash{}citeonline[p.$\sim$128]\{kop95\}.}}
\hspace{1em} ...que produz...\hspace{1em}
	{\parbox[t]{6.5cm}{%
	\footnotesize Uma explicação mais detalhada pode
	ser encontrada em  
	\citeonline[p.~128]{kop95}.}}\\[12pt]
Quando há três autores, o uso do comando 
\verb|\citeonline{|\emph{rótulo}\verb|}| produz:\\[6pt]
	{\parbox[t]{5.5cm}{%
	\ttfamily\footnotesize
	\textbackslash{}citeonline\{huey03\}
	discutem o fenômeno sob...}}
\hspace{1em} ...que produz...\hspace{1em}
	{\parbox[t]{6.5cm}{%
	\footnotesize 
	\citeonline{huey03}
	discutem o fenômeno sob...}}\\[12pt]
Quando há mais de três autores, o resultado é mostrado no exemplo
a seguir:\\[6pt]
	{\parbox[t]{5.5cm}{%
	\ttfamily\footnotesize
	\textbackslash{}citeonline\{huey05\}
	é a\\ referência básica do assunto.}}
\hspace{1em} ...que produz...\hspace{1em}
	{\parbox[t]{6.5cm}{%
	\footnotesize 
	\citeonline{huey05}
	é a referência básica do assunto.}}\\[12pt]

\section{A base de dados bibliográficos}

\textsc{Bib}\TeX\ é um programa associado ao \LaTeX\ que usa o pacote \texttt{abntex2cite.sty} e uma base de dados bibliográficos para
construir a bibliografia do documento. A base (ou bases) de dados usada para 
montar a lista de referências bibliográficas é indicada no documento
raiz pelo comando 
\texttt{\textbackslash{}bibliography\{\textrm{\emph{arq\_1,...,arq\_n}}\}},
em que \emph{arq\_i} é um arquivo com extensão \texttt{.bib}, que contém
as dados usados para a montagem das referências bibliográficas. 

Uma entrada do arquivo \emph{arq\_i}\texttt{.bib} tem o seguinte formato
geral:\\
\hspace*{3em}\verb|@tipo{rótulo,|\\
	\hspace*{5em}\verb|dados_obrigatórios [,|\\ 
	\hspace*{5em}\verb|dados_opcionais] }| \\
em que \verb|@tipo| define o tipo da referência 
   			bibliográfica (artigo, livro, tese etc.); 
	\texttt{rótulo} é uma string usada para 
			identificar a referência (como \texttt{setti98a});
	\texttt{dados\_obrigatórios} são os campos de dados que 
   			devem ser fornecidos para possibilitar a construção 
   			da referência pelo \textsc{Bib}\TeX; e
	\texttt{dados\_opcionais} são dados que complementam
     		os obrigatórios. 
Os tipos de referência e os dados, obrigatórios
e opcionais, são descritos no item~\ref{bibtex_entries}, a seguir.

É boa política preencher todos os campos de dados para os quais você
dispõe de informações. Você poderá reusar um arquivo \texttt{.bib} em
qualquer arquivo \texttt{.tex} que você fizer e alguns pacotes 
para montagem de bibliografias podem exigir certos dados além do
mínimo exigido pelo \textsc{Bib}\TeX. Por exemplo,  \texttt{abntex2cite.sty}
irá colocar \emph{[S.l.]} (sem local) em referências nas quais o campo 
\texttt{address} não existir (dependendo do tipo de referência), 
apesar desse campo ser opcional para a maioria dos tipos de referência
previstos no \textsc{Bib}\TeX.


O arquivo 
\texttt{bib/referencias.bib} contém exemplos de elementos bibliográficos
numa base de dados do \textsc{Bib}\TeX.{} Os tipos de referência previstos
pelo \textsc{Bib}\TeX\ incluem:
	\begin{alineas}
		\item livro (\verb|@book|),  \cite{kop95};
		\item capítulo ou trecho de livro (\verb|@inbook|);
		\item artigo em periódico (\verb|@article|) \cite{huey03};
		\item trabalho publicado em anais de congresso 
		(\verb|@inproceedings|);
		\item anais (completos) de congresso (\verb|@proceedings|);
		\item manual (\verb|@manual|) \cite{epslatex06};
		\item tese de doutorado (\verb|@phdthesis|);
		\item dissertação de mestrado (\verb|@mastersthesis|);
		\item relatório técnico (\verb|@techreport|); e
		\item documento que não pode ser classificado em
		nenhum outro tipo (\verb|@misc|).
	\end{alineas}
Além dos listados, que são os mais comuns numa tese ou
dissertação, há muitos outros elementos especificados
no manual do \texttt{abntex2cite.sty}, disponível neste 
\href{http://ctan.math.washington.edu/tex-archive/macros/latex/contrib/abntex2/doc/abntex2cite-alf.pdf}{\fbox{\texttt{link}}}.

O Google Scholar permite exportar referências em formato \textsc{Bib}\TeX,
o que pode facilitar um pouco sua vida, se o tipo e o formato estiverem
corretos (o que nem sempre acontece, na minha experiência).

\subsection{Tipos de referência e dados necessários}
\label{bibtex_entries}

%%----
% Definição de um novo comando, \campo para evitar escrever
%	\hspace*{1em}\texttt{_arg_}: em todos os campos
\newcommand{\campo}[1]{\hspace*{1em}\texttt{#1}:}
%%----
% Define comando \ttz{...} que equivale a \texttt{...}
\newcommand{\ttz}[1]{\texttt{#1}}

A lista a seguir define os tipos de referências previstos no 
\textsc{Bib}\TeX\ e, para cada um deles, quais são os dados 
obrigatórios e os opcionais, explicando como cada campo
deve ser preenchido. Os tipos podem ser escritos em minúsculas
\verb|@tipo| ou maiúsculas \verb|@TIPO|, sem causar problemas. 
Para maiores detalhes sobre os tipos de referências e os campos
de cada tipo consulte este 
\href{http://bib-it.sourceforge.net/help/fieldsAndEntryTypes.php}
{\fbox{\texttt{link}}}.
\begin{alineas}

	\item \verb|@BOOK| -- se a referência for um livro publicado por uma
		editora (use \verb|@inbook| para capítulo ou trecho de livro) \\
		\emph{Campos obrigatórios:}\\
 			\campo{author [ou] editor} nome do autor ou do editor;\\
 			\campo{title} título do livro;\\
 			\campo{publisher} nome da editora que publicou o livro; e \\
 			\campo{year} ano da publicação.\\
		\emph{Campos opcionais}:\\
			\campo{volume} número do volume, se houver mais de um; \\
			\campo{series} número do livro, se fizer parte de uma série;\\
 			\campo{address} cidade de publicação (evite omitir este campo);\\
 			\campo{edition} número da edição, se não for a única;\\
 			\campo{month} mês de publicação;\\
 			\campo{note} qualquer observação desejada; \\
 			\campo{key} usado para ordenação alfabética e citação, se
 				\texttt{author} e \texttt{editor} não existem.\\
		\emph{Exemplos em \texttt{bib/referencias.bib}:} \\
			\hspace*{1em}\citeonline{hay82} e \citeonline{kop95} ou 
			\cite{hay82,kop95,epe13,epe13b}\\
		\emph{Observações:} Não confundir \texttt{key}
 				com \texttt{rótulo}, que é usado para identificar a 
 				referência para o \textsc{Bib}\TeX. \texttt{key} não
 				é usado por \texttt{abntex2cite.sty}, por causa da 
 				norma da ABNT. Para livros sem autor e editor, eu sugiro 
 				verificar os dados de \citeonline{epe13}, para evitar
 				algo assim: \citeonline{epe13b} -- os dois são o
 				mesmo livro.
 				
	\item \verb|@INBOOK| -- capítulo ou parte de um livro (veja 
		\verb|@incollection| para outro caso de parte de livro) \\
		\emph{Campos obrigatórios:}\\
 			\campo{chapter [ou] pages} número do capítulo ou das páginas 
 			  citadas;\\
 			\texttt{author} ou \texttt[editor]; \texttt{title}; 
 			\texttt{publisher}; e \ttz{year}.\\
		\emph{Campos opcionais}:\\
			\texttt{volume}; \texttt{series}; \texttt{address};
			\texttt{edition};\texttt{month}; \texttt{note}\\
		\emph{Exemplos em \texttt{bib/referencias.bib}:} \\
			\hspace*{1em}\citeonline{kop95c3} ou 
			\cite{kop95c3,epe13a} 
	\emph{Observações:} O resultado fica ruim nos dois casos.
			Não dá para usar com o \texttt{abntex2cite.sty}.
		
	\item \verb|@INCOLLECTION| -- capítulo ou parte de um livro com
		seu próprio título e autor \\
		\emph{Campos obrigatórios:}\\
 			\campo{booktitle}  título do livro;\\
 			\campo{title} título do capítulo; \\
 			\campo{author} autor do capítulo; \\
 			\texttt{publisher}; e \ttz{year}.\\
		\emph{Campos opcionais}:\\
			\texttt{editor}; \texttt{volume [ou] número}; 
			\texttt{series}; \texttt{address};
			\texttt{edition};\texttt{month}; \texttt{note}\\
		\emph{Exemplos em \texttt{bib/referencias.bib}:} \\
			\hspace*{1em}\citeonline{cop05c2,cope2005} 
			\cite{cop05c2,cope2005}
		\emph{Observações:} O resultado fica ruim nos dois casos, 
			coloca-se o título do livro no lugar do título do 
			do capítulo. Não dá para usar com o \texttt{abntex2cite.sty}. 
			Veja como colocar sobrenomes compostos com \emph{Junior}, 
			\emph{Neto} e \emph{Filho} para que saiam corretamente
			no texto e na bibliografia. 

		\item \verb|@ARTICLE| para um artigo publicado num periódico
			científico ou numa revista\\
		\emph{Campos obrigatórios:}\\
 			\campo{journal}  nome do periódico;\\
 			\hspace*{1em}\ttz{title}; \ttz{author}; e \ttz{year}.\\
		\emph{Campos opcionais}:\\
			\hspace*{1em}\texttt{volume}; \texttt{number}; 
			\texttt{pages}; \texttt{month}; e \texttt{note}\\
		\emph{Exemplos em \texttt{bib/referencias.bib}:} \\
			\cite{huey03}
			
	\item \verb|@INPROCEEDINGS| trabalho publicado nos anais de
			um congresso científico\\
		\emph{Campos obrigatórios:}\\
 			\campo{booktitle}  título dos anais do congresso;\\
 			\hspace*{1em}\ttz{title}; \ttz{author}; e \ttz{year}.\\
		\emph{Campos opcionais}:\\
			 \hspace*{1em}\ttz{editor}; \ttz{volume}; \ttz{number}; \ttz{series}; 
			 \ttz{address}; 
			\texttt{pages}; \texttt{month}; \ttz{organization};\\
			\hspace*{1em}\ttz{publisher} e \texttt{note}\\
		\emph{Exemplos em \texttt{bib/referencias.bib}:} \\
			\cite{mil2009}, \citeonline{mil2009} e \cite{aut:06} \\
		\emph{Observações:} Os sobrenomes ``de Mileto'' e ``von Samos''
		são tratados corretamente se ``de'', ``von'' ou ``da'' estiverem
		em minúsculas. Use o campo \ttz{note} para incluir
		coisas como ``(CD-ROM)'' ou a url dos anais do congresso. 
		\citeonline{aut:06} é um exemplo de citação cruzada
		(veja \ttz{aut:06} e \ttz{conf:06} no arquivo \ttz{.bib}). 
			
			
	\item \verb|@PROCEEDINGS| para citar os anais de um congresso
			científico (o livro todo)\\
		\emph{Campos obrigatórios:}\\
 			\hspace*{1em}\ttz{title} e \ttz{year}.\\
		\emph{Campos opcionais}:\\
			 \hspace*{1em}\ttz{editor}; \ttz{volume}; \ttz{number}; 
			 \ttz{series}; \ttz{address}; \texttt{month}; \ttz{organization};\\
			\hspace*{1em}\ttz{publisher} e \texttt{note}\\
		\emph{Exemplos em \texttt{bib/referencias.bib}:} \\
			\citeonline{conf:06} e \cite{conf:06} \\
			
	\item \verb|@MANUAL| normalmente, manual de um software (veja
		\ttz{techreport} \\
		\emph{Campos obrigatórios:}\\
 			\hspace*{1em}\ttz{title}.\\
		\emph{Campos opcionais}:\\
			 \hspace*{1em}\ttz{author};   \ttz{organization};
			 \ttz{edition}; \ttz{address}; \texttt{month};
			\ttz{year} e \texttt{note}\\
		\emph{Exemplos em \texttt{bib/referencias.bib}:} \\
				\citeonline{epslatex06}, \cite{epslatex06}

	\item \verb|@TECHREPORT| um relatório técnico, publicado por
		uma organização, com ou sem autor, às vezes numerado numa
		série \\
		\emph{Campos obrigatórios:}\\
 			\hspace*{1em}\ttz{title}; \ttz{author}; \ttz{institution}
 			e \ttz{year}\\
		\emph{Campos opcionais}:\\
			 \hspace*{1em}\ttz{type};   \ttz{number};
			 \ttz{address}; \texttt{month};
			e \texttt{note}\\
		\emph{Exemplos em \texttt{bib/referencias.bib}:} \\
				\citeonline{gomez03} (\ttz{gomez03}), \cite{dnit06}
				(\ttz{dnit06})
		\emph{Observações:} Veja que não aparece nem o tipo de relatório
		nem seu número -- mais um problema do \ttz{abntex2cite.sty}. Em 
		\ttz{gomez03} não aparece o nome da instituição. Melhor evitar
		este tipo no \ttz{abntex2cite.sty}.	Tente \ttz{misc}.		
								
	\item \verb|@PHDTHESIS| uma tese de doutorado \\
		\emph{Campos obrigatórios:}\\
 			\hspace*{1em}\ttz{author}; \ttz{title}; \ttz{school}; e \ttz{year}.\\
		\emph{Campos opcionais}:\\
			 \hspace*{1em}  \ttz{type};
			 \ttz{address}; \texttt{month};
			 e \texttt{note}\\
		\emph{Exemplos em \texttt{bib/referencias.bib}:} \\
				\citeonline{cunha13}, \cite{cunha13}
	
	\item \verb|@MASTERSTHESIS| uma dissertação de mestrado \\
		\emph{Campos obrigatórios:}\\
 			\hspace*{1em}\ttz{author}; \ttz{title}; \ttz{school}; e \ttz{year}.\\
		\emph{Campos opcionais}:\\
			 \hspace*{1em}  \ttz{type};
			 \ttz{address}; \texttt{month};
			 e \texttt{note}\\
		\emph{Exemplos em \texttt{bib/referencias.bib}:} \\
				\citeonline{cunha09}, \cite{cunha09}
					
	\item \verb|@misc| um documento genérico, que pode resolver
		alguns dos seus problemas \\
		\emph{Campos obrigatórios:}\\
 			\hspace*{1em}nenhum\\
		\emph{Campos opcionais}:\\
			 \hspace*{1em}\ttz{author}; \ttz{title}; 
			 \ttz{howpublished}; \texttt{month}; \ttz{year};
			 e \texttt{note}\\
		\emph{Exemplos em \texttt{bib/referencias.bib}:} \\
				\citeonline{oli87} e \cite{oli87} e, para um documento na
				web: \cite{epsl}. Teste para ver se fica certo 
				\cite{epsl, epslatex06}.
			
\end{alineas}


