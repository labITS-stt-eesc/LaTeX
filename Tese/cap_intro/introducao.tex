% !TEX encoding = ISO-8859-1	
% Define o encoding do arquivo (TeXworks e TeXstudio?) para que os acentos possam ser
% usados diretamente: é e não \'e

\chapter{Introdução}

Este documento que simula parte de uma tese de doutorado foi elaborado para facilitar o uso da
classe \texttt{\abnTeX.cls}\ na elaboração de teses e dissertações no \textbf{Programa de Pós-Graduação
em Engenharia de Transportes}, da EESC-USP, usando a classe customizada \texttt{eesc-stt.cls}. 

Esta customização para o PPG-ET foi criada a partir da classe \texttt{eesc.cls},
por sua vez também uma customização da classe \texttt{\abnTeX.cls}, feita por Athila Quaresma Santos e Renato Monaro,   
para atender às exigências da CPG-EESC (mais especificamente, do Programa de 
Pós-Graduação em Engenharia Elétrica e de Computação da EESC). 

A classe \texttt{\abnTeX.cls} é uma adaptação da classe \texttt{memoir.cls} para as 
(muitas vezes inexplicáveis) exigências da NBR-14724:2011 \emph{Informação 
e documentação -- Trabalhos acadêmicos -- Apresentação} e suas diversas ``irmãs''. A classe \texttt{\abnTeX.cls} requer o uso do 
o pacote \texttt{abntex2cite.sty} para gerar a
bibliografia e as citações bibliográficas ao longo do texto em conformidade
com a norma NBR-10520:2002 \cite{NBR10520:2002}.
 
Lendo este documento e os comentários colocados ao longo do seu código-fonte, 
você poderá familiarizar-se com o uso da classe \texttt{eesc-stt.cls} e do pacote 
\texttt{abntex2cite} e aprender como usar o \LaTeX\ em algumas situações comuns
durante a elaboração de uma tese/dissertação. Em pontos apropriados do texto, há
links clicáveis para documentação mais completa ou, pelo menos, indicação da 
sua existência no CTAN (Comprehensive \TeX\ Archive Network) -- 
\href{http://www.ctan.org/pkg/abntex2}{\fbox{link para CTAN.org}}.

Você pode gerar este pdf processando o arquivo \texttt{tese-stt.tex} no \LaTeX.